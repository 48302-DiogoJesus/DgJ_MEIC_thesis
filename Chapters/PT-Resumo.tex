% #############################################################################
% RESUMO em Português
% !TEX root = ../main.tex
% #############################################################################
% reset acronyms
\acresetall
% use \noindent in firts paragraph
\noindent A computação serverless tornou-se um paradigma de nuvem adequado para muitas aplicações, valorizado pela sua facilidade operacional, escalabilidade automática e modelo de preços granular baseado na utilização. Contudo, a execução de workflows, que são composições de múltiplas tarefas, em ambientes Function-as-a-Service (FaaS) permanece ineficiente. Esta ineficiência resulta da natureza \textit{stateless} (sem estado) destas funções e de uma forte dependência de serviços externos para transferências de dados intermédios e comunicação entre funções.

Neste documento, apresentamos um motor de workflows serverless descentralizado que utiliza métricas recolhidas durante a execução para planear e influenciar o \textit{scheduling} de tarefas. A nossa solução abrange a gestão de metadados, o planeamento estático de workflows e uma estratégia de \textit{scheduling} ao nível dos workers concebida para conduzir a execução de workflows de uma forma descentralizada e com sincronização mínima. Comparamos a nossa abordagem com o WUKONG, outro motor de workflows serverless descentralizado. A nossa avaliação demonstra que a utilização de informação histórica melhora significativamente o desempenho e reduz a utilização de recursos para workflows executados em plataformas serverless.